
\documentclass[10pt, a4paper, titlepage]{article}

\usepackage[utf8]{inputenc}				% napsáno v UTF-8
\usepackage[tables,figures]{upreport}	% UPOL styl
\usepackage{upstyles}
\usepackage{amsmath}
\usepackage{amssymb}
\usepackage{amsthm}
\usepackage{alltt}


\setlength{\parindent}{0pt}
\setlength{\parskip}{1ex plus 0.5ex minus 0.2ex}

\title{KMI/NLO  -- Neklasické logiky}
\author{\normalsize{Radek Janoštík (radek.janostik01@upol.cz)}}
\date{\today}

\docinfo{Radek Janoštík}{KMI/NLO  -- Neklasické logiky}

\newtheoremstyle{note}
{15pt}
{15pt}
{}
{}
{\bfseries}
{:}
{.5em}
{}

\theoremstyle{note}
\newtheorem{veta}{Věta}
\newtheorem{definice}{Definice}
\newtheorem{priklad}{Příklad}
\newtheorem{poznamka}{Poznámka}
\newtheorem{dusledek}{Důsledek}

\newcommand{\pk}{\text{ p.k. }}
\newcommand{\rlat}{\langle L, \wedge, \vee,\otimes,\rightarrow,0,1\rangle}

\abstract{
Tento dokument je pouze přepisem zápisků a poznámek
z přednášek předmětu KMI/NLO. Přednášel \link{doc. Vilém Vychodil PhD}{http://vychodil.inf.upol.cz/}.}

\pagestyle{empty}
\makeindex

\begin{document}
\maketitle
\section{Přednáška 1 - jemný úvod}
\subsection{Formální logika} - studium vyplývání $\rightarrow$ formalizuje výroky, výrazy přirozeného jazyka $\rightarrow$ formule. Definuje se, že formule je/není důsledkem jiných formulí.

\subsection{Odlišnosti logik}
\begin{enumerate}
\item Co vše popisuje jazyk - tj. co jsme schopni vyjádřit pomocí formulí.

  Př.:
  \begin{enumerate}
    \item Výroková logika - zabývá se výroky - neformálně výraz, o kterém se uvažuje, že je pravdivý či ne. 

      Atomická formule - nemůže se dělit na podvýrazy pomocí spojek. Nahrazují je výrokové symboly

      Složitější formule - 1)Výrokový symbol je formule.

			   2) Je-li $\varphi$ formule, pak i $\neg \varphi$ je formule.

			   3) Jsou-li $\varphi ,\psi$ formule, pak i $\varphi \Rightarrow \psi$ je formule.
  \item Predikátová logika - zabývá se (mj.) strukturou výroků

      $(\forall x)(\forall y) (x \leq y \Rightarrow f(x) \leq f(y))$ - formule jazyka, kde $R = \{\leq\}$, $F = \{f\}$

  \item Modální logiky - formalizují modality - "muset", "moci" \dots

      Modální výroková - $\square$ \dots musí, $\diamond$ \dots může.

	Formule: Je-li $\varphi$ formule, pak i $\square\varphi$ a $\diamond\varphi$ jsou formule.

  Paradox Arnošta Večerky: \uv{Když mám 10 korun, koupím si čokoládu.}:  $\varphi \Rightarrow\psi$

  \uv{Když mám 10 korun, koupím si bonbon.}: $\varphi \Rightarrow \chi$

  $T=\{\varphi\Rightarrow\psi , \varphi\Rightarrow\chi\}$  $T \vdash \varphi\Rightarrow(\psi \wedge \chi)$
    
Modální logika dodá \uv{může}.
  $T=\{\varphi\Rightarrow\diamond\psi,\varphi\Rightarrow\diamond\chi\}$ $T\vdash\varphi\Rightarrow(\diamond\psi\wedge\diamond\chi)$. Pozor: $T \not\vdash \varphi\Rightarrow\diamond (\psi\wedge\chi)$
  \end{enumerate}

\item Tím, jak zavádí vyplývání
  \begin{enumerate}
    \item Sémantické - navrhneme interpretaci formulí.

    VL: zavedeme ohodnocení: $e: V\rightarrow\{0,1\}$ $||\varphi||_{e}$ \dots

    PL: $\langle R, F, \sigma\rangle \rightarrow \mathbb{M}=\langle M, R^{M}, F^{M}\rangle$ $||\varphi||_{M,v}$ \dots $T \models \varphi$

    mod. VL: $\square\varphi , \diamond\varphi$ - Kripkeho struktura - $\mathbb{K} = \langle W, r, e\rangle$ 

  $r\subseteq W \times W$ $\langle w_{1},w_{2}\rangle \in r$ \dots $w_{2}$ je dosažitelný z $w_{1}$ 

$e: W \times V \rightarrow \{0,1\}$

$||\square\varphi||_{\mathbb{K},w} = 1$ pokud \underline{pro každý} $w'\in W$ platí: pokud $\langle w, w'\rangle\in r$ pak $||\varphi||_{\mathbb{K},w'}=1$

$||\diamond\varphi||_{\mathbb{K},w} = 1$ \dots existuje \dots

  \item Syntaktické \dots důkaz

  kl. VL: Pravidlo: z $\varphi, \varphi\Rightarrow\psi$ odvoď $\psi$

    $(Ax): \varphi\Rightarrow(\psi\Rightarrow\varphi)$

    $(\varphi\Rightarrow(\varphi\Rightarrow\chi))\Rightarrow((\varphi\Rightarrow\psi)\Rightarrow(\varphi\Rightarrow\chi))$

    $(\neg\psi\Rightarrow\neg\varphi)\Rightarrow(\varphi\Rightarrow\psi)$

    PL: Pravidlo: z $\varphi$ odvoď $(\forall x)\varphi$

    Distrib: $(\forall x)\varphi\Rightarrow\varphi(x/t)$

    Spec: $(\forall x)(\varphi\Rightarrow\psi)\Rightarrow(\varphi\Rightarrow(\forall x) \psi)$

  \end{enumerate}

  Lze zavést vyplývání jinak? - Alternativní syntaktické vyplývání - Gentzenovské dokazovací systémy - \uv{natural deduction}

  Např.: V.Vychodil - prahová booleovská logika.
\end{enumerate}

\subsection{Co budeme zkoumat tento semestr}
Neklasické logiky, ve kterých se uvažuje, že atomické formule mohou nabývat \underline{stupňů pravdivosti} - $0$\dots$1$ - mezní dva stupně.

  $0$\dots (plně) nepravdivý

  $1$\dots (plně) pravdivý

  $0 < a < 1$ \dots stupeň pravdivosti

$\rightarrow$ \underline{Základní interpretace} $\rightarrow$ komparativní

$||\varphi||_{a} = a, ||\psi||_{b}=b$ $a \leq n$ \dots $\varphi$ je méně pravdivá než $\psi$.

$e: V\rightarrow L$ $\mathbb{L} = \langle L, \leq, 0, 1\rangle$ \dots ohraničená uspořádaná množina.

$||\varphi||_{e} \in L$

$\mathbb{L} = \langle L, \wedge, \vee, 0, 1\rangle$ \dots úplný svaz (zbytečně silné)

$\mathbb{L} = \langle L, \wedge, \vee, 0, 1\rangle$ \dots svaz (ohraničený)

Zbývá vyřešit, jak interpretovat logické spojky a které spojky vzít jako základní.

Princip kompozicionality: $||\varphi\Rightarrow\psi||_{e} = ||\varphi||_{e}\rightarrow||\psi||_{e}$

$\rightarrow$ - logická operace, která interpretuje $\Rightarrow$: 

$\rightarrow : L\times L \rightarrow L$:
 \begin{tabular}{l|*2c}
     \bf $\rightarrow$ & 0 & 1  \\
     \hline
    0 & 1& 1 \\
    1 & 0 & 1 \\
   \end{tabular}

60. léta Lotfi Askerzadeh - koncept fuzzy množiny - $\wedge\dots\min$, $\vee\dots\max$, $\neg \dots 1-a$

J.A.Goguen - $\mathbb{L} = \langle L, \wedge, \vee, 0, 1\rangle$

\underline{Modus ponens} - $\frac{\varphi\Rightarrow\psi,\varphi}{\psi}$ - 1. pohled $\frac{||\varphi\Rightarrow\psi||_{e}=1,||\varphi||_{e} = 1}{||\psi||_{e}=1}$

2. pohled - min. dolní mez pravdivosti $\psi$ odvozujeme z min. dolních mezí předchozích dvou. $\frac{a\leq||\varphi\Rightarrow\psi||, b\leq ||\varphi||}{a\otimes b\leq||\psi||}$ \dots$\otimes\rightarrow\{0,1\}\dots$ pravdivostní funkce konjunkce.

\underline{Pozorování}: $\Rightarrow\dots||\varphi\Rightarrow\psi|| = ||\varphi||\rightarrow||\psi||$

  $\otimes\dots$ pravdivostní funkce konjunkce

Jaký by měly mít vztah $(\otimes,\rightarrow)$ ? Chceme, aby zobecněné MP bylo korektní:

Pokud $a\leq||\varphi\Rightarrow\psi||$ a $b\leq||\varphi||$ pak $a\otimes b \leq||\psi||$ pro $b=||\varphi||$ a $c=||\psi||$ použitím $||\varphi\Rightarrow\psi||=||\varphi||\rightarrow||\psi||=b\rightarrow c$

Pokud $a\leq||\varphi\Rightarrow\psi|| = b\rightarrow c$, pak $a\otimes b\leq c$

$a\leq b\rightarrow c$ pak $a\otimes b \leq c$

Zobecněné MP mělo maximální možnou sílu - $b=||\varphi||, c=||\psi||$:

Pokud $a\otimes b\leq||\psi||=c$, pak $a\leq||\varphi||\rightarrow||\psi||=b\rightarrow c$

Dohromady: $$a\otimes b\leq c \pk a\leq b\rightarrow c \dots \text{ adjunkce}$$

(Úplný) reziduovaný svaz $\mathbb{L} = \rlat$

$\langle L, \wedge,\vee, 0, 1\rangle \dots$(úplný) svaz

$\langle L, \otimes, 1\rangle\dots\otimes\dots$binární operace, komutativní, asociativní, $a\otimes 1=a$

$\rightarrow\dots$ binární operace, která splňuje adjunkci.

\vspace{1cm}
MP: $\frac{\varphi, \neg\varphi\vee\psi}{\psi} \rightarrow \frac{\varphi\vee\psi, \neg\varphi}{\psi}$ skrytý sém. význam: $\frac{1\leq||\varphi\vee\psi||, ||\varphi||\leq 0}{1\leq||\psi||}$

$\frac{a\leq||\varphi\vee\psi||, ||\varphi||\leq b}{c\leq||\psi||}$, \hspace{2cm} $c=a\ominus b$ \hspace{2cm}   \begin{tabular}{l|*2c}
     \bf $\ominus$ & 0 & 1  \\
     \hline
    0 & 0& 0 \\
    1 & 1 & 0 \\
   \end{tabular} $a \ominus b = \neg(a\rightarrow b)$

$\vee\dots\oplus$\hspace{3cm} $\ominus\dots$ pravdivostní funkce abjunkce - \uv{neimplikace}

Pro $b = ||\varphi||, c=||\psi||$ korektnost: $a\leq||\varphi\vee\psi|| = b\oplus c$ pak $a\ominus b\leq c$

síla: $a\ominus b \leq ||\psi||=c$ pak $a\leq b\oplus c$

$\oplus\dots$ komutativní, asociativní, neutrální vůči 0

$\ominus\dots$ adjungovaná k $\oplus$

$$a\ominus b\leq c \text{ p. k. } a\leq b \oplus c$$

\section{Přednáška 2 - reziduované svazy}
$\mathbb{L} = \rlat$

$\langle L, \wedge,\vee, 0, 1\rangle \dots$ -- ohraničený nebo úplný svaz

$\leq \dots a\leq b \pk a \wedge b = a \pk a \vee b = b$

$\langle L, \otimes, 1\rangle\dots$ komutativní monoid (= asociativní, neutrální prvek -- 1)

$\rightarrow\dots$ binární operace: $$a\otimes b\leq c \pk a\leq b\rightarrow c \text{ pro } \forall a,b,c$$

\begin{veta}
$a\otimes 0 = 0\otimes a$ tj. $\otimes$ se na hodnotách z $\{0,1\}$ chová stejně jako pravdivostní funkce klasické konjunkce.
\end{veta}
\begin{proof}
$0\leq a \rightarrow 0$, protože $0$ je nejmenší z $L$.

Z adjunkce: $0\otimes a\leq 0$ tj. $0\otimes a=0$. Opačná nerovnost plyne z komutativity $\otimes$
\end{proof}
\begin{veta}
\label{v2}
$a\rightarrow b = 1 \pk a\leq b$

$1\rightarrow 0 = 0$ tj. $\rightarrow$ se na hodnotách $\{0,1\}$ chová jako klasická implikace.
\end{veta}
\begin{proof}
$a\leq b \pk a\otimes 1\leq b\pk 1\leq a\rightarrow b$ ($\geq$ platí vždy - 1 je největší prvek)

$1\rightarrow 0\leq 1\rightarrow 0$

$1\otimes (1\rightarrow 0)\leq 0$ z neutrality $1\rightarrow 0\leq 0$ ($\geq$ platí vždy - 0 je nejmenší prvek.)
\end{proof}
\begin{veta}
Reziduum $\rightarrow$ je jednoznačně dané součinem $\otimes$ a obráceně. (Ale nemusí vůbec existovat)
\end{veta}
\begin{proof}
Nechť $\langle \otimes ,\rightarrow_{1}\rangle$ a $\langle \otimes ,\rightarrow_{2}\rangle$ jsou odjungované páry.

$a\leq b\rightarrow_{1} c\pk a\otimes b\leq c\pk a\leq b\rightarrow_{2} c$

Dále jasné: pro $a=b\rightarrow_{1} c$ vyplývá $b\rightarrow_{1} c\leq b\rightarrow_{2} c$

\hspace{51pt}pro $a=b\rightarrow_{2} c$ vyplývá $b\rightarrow_{2} c\leq b\rightarrow_{1} c$
\end{proof}

\begin{priklad}
Todo: Obrázek diamantu. Diamant $\mathbb{N}_{3}$ -- neexistuje žádný adjungovaný pár.
\end{priklad}
\begin{priklad}
Todo: Obrázek 4hodnotové boolovy algebry. $x\otimes y=x\wedge y$

$x\rightarrow y = x\vee y$ \dots 4hodnotová booleova algebra.
\end{priklad}
\begin{veta}
\begin{enumerate}
\renewcommand{\labelenumi}{(\alph{enumi})}
\item $a\leq b \rightarrow(a\otimes b)$ \dots $a\otimes b\leq a\otimes b$
\item $b\leq a \rightarrow(a\otimes b$ \dots z komutativity $\otimes$
\item $a\otimes (a\rightarrow b) \leq b$ \dots $a\rightarrow b\leq a\rightarrow b$
\item $a\leq (a\rightarrow b) \rightarrow b$
\end{enumerate}
\end{veta}

\begin{dusledek}
Tato věta + Věta \ref{v2}: $a\rightarrow (b\rightarrow (a\otimes b)) = 1$

\hspace{5cm} $(a\otimes (a\rightarrow b))\rightarrow b=1$
\end{dusledek}
\begin{veta}
$a\rightarrow (b\rightarrow c) = (a\otimes b)\rightarrow c = b\rightarrow(a\rightarrow c)$
\end{veta}
\begin{proof}
\begin{enumerate}
\item \begin{eqnarray}
  a\rightarrow(b\rightarrow c)&\leq& a\rightarrow(b\rightarrow c) \nonumber\\
  a\otimes(a\rightarrow(b\rightarrow c))&\leq& b\rightarrow c \nonumber\\
  b\otimes a\otimes(a\rightarrow(b\rightarrow c))&\leq& c \nonumber\\
  (a\otimes b)\otimes(a\rightarrow(b\rightarrow c))&\leq& c \nonumber\\
  a\rightarrow(b\rightarrow c)&\leq& (a\otimes b)\rightarrow c\nonumber
  \end{eqnarray}

\item \begin{eqnarray}
  (a\otimes b) \rightarrow c &\leq &(a\otimes b) \rightarrow c \nonumber\\
  (a\otimes b)\otimes((a\otimes b) \rightarrow c) &\leq &  c \nonumber\\
a\otimes((a\otimes b) \rightarrow c) &\leq & b\rightarrow c \nonumber\\
(a\otimes b) \rightarrow c &\leq & a \rightarrow (b\rightarrow c) \nonumber
  \end{eqnarray}
  Dokázána první rovnost, druhá plyne z komutativity $\otimes$.
\end{enumerate}
\end{proof}
\begin{veta}
$\otimes$ je izotonní v obou argumentech. Pokud $a\leq b$ pak $a\otimes c\leq b\otimes c$. Tedy pokud $a_{1}\leq b_{1}$ a $a_{2}\leq b_{2}$ pak $a_{1}\otimes a_{2}\leq b_{1}\otimes b_{2}$
\end{veta}
\begin{proof}
\begin{eqnarray}
  b\otimes c &\leq&b\otimes c \nonumber\\
  b &\leq& c\rightarrow (b\otimes c) \nonumber\\
  \text{tj.:} a&\leq& c\rightarrow(b\otimes c) \text{  // z transitivity} \leq \nonumber\\
  a\otimes c &\leq&b\otimes c\nonumber
  \end{eqnarray}
Druhý bod plyne dvojnásobným použitím předchozího: $a_{1}\otimes a_{2}\leq b_{1}\otimes a_{2}\leq b_{1}\otimes b_{2}$
\end{proof}

\begin{veta}
$(a\rightarrow b) \otimes (b\rightarrow c)\leq a\rightarrow c$
\end{veta}
\begin{proof}
Stačí ukázat, že platí: \begin{eqnarray} a\otimes(a\rightarrow b)\otimes(b\rightarrow c)&\leq& c\nonumber\\
	      a\otimes(a\rightarrow b)&\leq&b\nonumber\\
	      a\otimes(a\rightarrow b)\otimes(b\rightarrow c)&\leq&b\otimes(b\rightarrow c) \leq c\nonumber
  \end{eqnarray}
\end{proof}
\begin{veta}
$\rightarrow$ je antitonní v prvním argumentu a izotonní v druhém argumentu. Tj. pokud $a\leq b$ pak $b\rightarrow c \leq a\rightarrow c$, pokud $b\leq c$ pak $a\rightarrow b\leq a\rightarrow c$
\end{veta}
\begin{proof}
Pokud $a\leq b$ z \begin{eqnarray} b\rightarrow c&\leq& b\rightarrow c\nonumber\\
				 b\otimes (b\rightarrow c)&\leq& c\nonumber\\
				 b &\leq& (b\rightarrow c)\rightarrow c\nonumber\\
				 a &\leq& (b\rightarrow c)\rightarrow c\nonumber\\
				 (b\rightarrow c)\otimes a &\leq&  c\nonumber\\
				 b\rightarrow c &\leq&  a \rightarrow c\nonumber
\end{eqnarray}
Pokud $b\leq c$ z:\begin{eqnarray}a\rightarrow b &\leq&a\rightarrow b \nonumber\\
				  a\otimes (a\rightarrow b) &\leq& b \nonumber\\
				  a\otimes (a\rightarrow b) &\leq& c \nonumber\\
				  (a\rightarrow b) &\leq& a\rightarrow c \nonumber
\end{eqnarray}
\end{proof}
\begin{veta}
$a\rightarrow b$ je největší prvek množiny $\{c\in L | a\otimes c\leq b\}$

\hspace{1.45cm}$a\otimes b$ je nejmenší prvek množiny $\{c\in L | a\leq b\rightarrow c\}$
\end{veta}
\begin{proof}
\begin{enumerate}
\item $a\rightarrow b$ patří do $\{a\otimes c\leq b\}$ protože $a\otimes(a\rightarrow b)\leq b$.
  Nechť $a\otimes c \leq b\dots$ z adjunkce $\dots c\leq a\rightarrow b$

\item Analogicky: $a\otimes b \in \{ c| a\leq b\rightarrow c\}$ protože $a\leq b\rightarrow(a\otimes b)$

Pokud $a\leq b\rightarrow c$ pak $a\otimes b\leq c$
\end{enumerate}
\end{proof}
\begin{veta}
\begin{eqnarray}
a\otimes \bigvee b_{i} &=&\bigvee(a\otimes b_{i}) \nonumber\\
a\rightarrow\bigwedge b_{i} &=&\bigwedge(a\rightarrow b_{i})\nonumber\\
\bigvee a_{i} \rightarrow b &=&\bigwedge(a_{i}\rightarrow b) \nonumber
\end{eqnarray}
-- 1. řádek: součin je distributivní přes $\bigvee$
\begin{proof}
\begin{enumerate}
\item Z monotonie $\otimes$: $a\otimes \bigvee_{i} b_{i}  \geq a\otimes b_{i}$ (druhé $i$ je zvolené) platí pro $\forall i$ tj. $a\otimes\bigvee b_{i} \geq \bigvee(a\otimes b_{i})$

Pravá strana:\begin{eqnarray}
a\otimes b_{i}&\leq& \bigvee(a\otimes b_{i})\nonumber \\
b_{i}&\leq&a\rightarrow\bigvee(a\otimes b_{i}) \text{ pro každé }i\nonumber \\
\bigvee b_{i}&\leq&a\rightarrow\bigvee(a\otimes b_{i})\nonumber\\
a\otimes \bigvee b_{i}&\leq&\bigvee(a\otimes b_{i})\nonumber
\end{eqnarray}
\item $a\rightarrow \bigwedge{i} \leq a\rightarrow b_{i}$ plat pro $\forall i$, tedy $a\rightarrow \bigwedge b_{i} \leq \bigwedge(a\rightarrow b_{i})$
\begin{eqnarray}
\bigwedge(a\rightarrow b_{i})&\leq& a\rightarrow b_{i} \text{ pro } \forall i\nonumber\\
a\otimes \bigwedge(a\rightarrow b_{i})&\leq& b_{i} \text{ pro } \forall i\nonumber\\
a\otimes \bigwedge(a\rightarrow b_{i})&\leq& \bigwedge b_{i} \nonumber\\
\bigwedge(a\rightarrow b_{i})&\leq& a\rightarrow \bigwedge b_{i} \nonumber
\end{eqnarray}
\item \begin{eqnarray}
\bigvee a_{i} \rightarrow b&\leq&a_{i}\rightarrow b \text{ pro } \forall i\nonumber\\
\bigvee a_{i} \rightarrow b&\leq& \bigwedge(a_{i}\rightarrow b) \nonumber\\\nonumber\\
\bigwedge(a_{i}\rightarrow b)&\leq& a_{i}\rightarrow b\nonumber\\
a_{i}\otimes \bigwedge(a_{i}\rightarrow b)&\leq&b\nonumber\\
a_{i} &\leq&(\bigwedge(a_{i}\rightarrow b)) \rightarrow b \text{ pro každé }i \nonumber\\
\bigvee a_{i} &\leq&(\bigwedge(a_{i}\rightarrow b)) \rightarrow b \nonumber\\
\bigwedge(a_{i}\rightarrow b) &\leq&\bigvee a_{i}\rightarrow b\text{   (dvojí adjunkce)} \nonumber
\end{eqnarray}
\end{enumerate}
\end{proof}
\end{veta}
\begin{poznamka}
$\vdash (\forall x)(\varphi\Rightarrow\psi)\Leftrightarrow(\varphi\Rightarrow(\forall x)\psi)$ -- protějšek 2.\footnote{Vůbec netuším, jak je to myšleno!}
\end{poznamka}
\begin{veta}
\begin{eqnarray}
a\otimes \bigwedge b_{i}&\leq&\bigwedge(a\otimes b_{i})\nonumber\\
\bigvee(a\rightarrow b_{i})&\leq& a\rightarrow \bigvee b_{i}\nonumber\\
\bigvee(a_{i}\rightarrow b)&\leq& \bigwedge a_{i}\rightarrow b\nonumber\\
\bigwedge(a_{i}\rightarrow b_{i})&\leq&\bigwedge a_{i}\rightarrow\bigwedge b_{i}\nonumber\\
\bigwedge(a_{i}\rightarrow b_{i})&\leq&\bigvee a_{i}\rightarrow \bigvee b_{i}\nonumber
\end{eqnarray}
\begin{proof}
1-3 \uv{jednoduché}, 4: $\bigvee a_{i} \otimes \bigwedge(a_{i}\rightarrow b_{i})\leq a_{i}\otimes(a_{i}\rightarrow b_{i})\leq \bigwedge b_{i}$

5: $\bigvee a_{i}\leq(a_{i}\rightarrow b_{i})\rightarrow b_{i}\leq \bigwedge(a_{i}\rightarrow b_{i})\rightarrow\bigvee b_{i}$
\end{proof}
\end{veta}
\renewcommand{\indexcolumns}{3}
\printindex
\end{document}