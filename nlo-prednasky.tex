
\documentclass[10pt, a4paper, titlepage]{article}

\usepackage[utf8]{inputenc}				% napsáno v UTF-8
\usepackage[tables,figures]{upreport}	% UPOL styl
\usepackage{upstyles}
\usepackage{amsmath}
\usepackage{amssymb}
\usepackage{amsthm}
\usepackage{alltt}


\setlength{\parindent}{0pt}
\setlength{\parskip}{1ex plus 0.5ex minus 0.2ex}

\title{KMI/NLO  -- Neklasické logiky}
\author{\normalsize{Radek Janoštík (radek.janostik01@upol.cz)}}
\date{\today}

\docinfo{Radek Janoštík}{KMI/NLO  -- Neklasické logiky}

\newtheoremstyle{note}
{15pt}
{15pt}
{}
{}
{\bfseries}
{:}
{.5em}
{}
\theoremstyle{note}
\newtheorem{dukaz}{Důkaz}
\newtheorem{veta}{Věta}
\newtheorem{definice}{Definice}
\newtheorem{priklad}{Příklad}
\newtheorem{poznamka}{Poznámka}
\newtheorem{dusledek}{Důsledek}


\abstract{
Tento dokument je pouze přepisem zápisků a poznámek
z přednášek předmětu KMI/NLO. Přednášel \link{doc. Vilém Vychodil PhD}{http://vychodil.inf.upol.cz/}.}

\pagestyle{empty}
\makeindex

\begin{document}
\maketitle
\section{Přednáška 1 - jemný úvod}
\subsection{Formální logika} - studium vyplývání $\rightarrow$ formalizuje výroky, výrazy přirozeného jazyka $\rightarrow$ formule. Definuje se, že formule je/není důsledkem jiných formulí.

\subsection{Odlišnosti logik}
\begin{enumerate}
\item Co vše popisuje jazyk - tj. co jsme schopni vyjádřit pomocí formulí.

  Př.:
  \begin{enumerate}
    \item Výroková logika - zabývá se výroky - neformálně výraz, o kterém se uvažuje, že je pravdivý či ne. 

      Atomická formule - nemůže se dělit na podvýrazy pomocí spojek. Nahrazují je výrokové symboly

      Složitější formule - 1)Výrokový symbol je formule.

			   2) Je-li $\varphi$ formule, pak i $\neg \varphi$ je formule.

			   3) Jsou-li $\varphi ,\psi$ formule, pak i $\varphi \Rightarrow \psi$ je formule.
  \item Predikátová logika - zabývá se (mj.) strukturou výroků

      $(\forall x)(\forall y) (x \leq y \Rightarrow f(x) \leq f(y))$ - formule jazyka, kde $R = \{\leq\}$, $F = \{f\}$

  \item Modální logiky - formalizují modality - "muset", "moci" \dots

      Modální výroková - $\square$ \dots musí, $\diamond$ \dots může.

	Formule: Je-li $\varphi$ formule, pak i $\square\varphi$ a $\diamond\varphi$ jsou formule.

  Paradox Arnošta Večerky: \uv{Když mám 10 korun, koupím si čokoládu.}:  $\varphi \Rightarrow\psi$

  \uv{Když mám 10 korun, koupím si bonbon.}: $\varphi \Rightarrow \chi$

  $T=\{\varphi\Rightarrow\psi , \varphi\Rightarrow\chi\}$  $T \vdash \varphi\Rightarrow(\psi \wedge \chi)$
    
Modální logika dodá \uv{může}.
  $T=\{\varphi\Rightarrow\diamond\psi,\varphi\Rightarrow\diamond\chi\}$ $T\vdash\varphi\Rightarrow(\diamond\psi\wedge\diamond\chi)$. Pozor: $T \not\vdash \varphi\Rightarrow\diamond (\psi\wedge\chi)$
  \end{enumerate}

\item Tím, jak zavádí vyplývání
  \begin{enumerate}
    \item Sémantické - navrhneme interpretaci formulí.

    VL: zavedeme ohodnocení: $e: V\rightarrow\{0,1\}$ $||\varphi||_{e}$ \dots

    PL: $\langle R, F, \sigma\rangle \rightarrow \mathbb{M}=\langle M, R^{M}, F^{M}\rangle$ $||\varphi||_{M,v}$ \dots $T \models \varphi$

    mod. VL: $\square\varphi , \diamond\varphi$ - Kripkeho struktura - $\mathbb{K} = \langle W, r, e\rangle$ 

  $r\subseteq W \times W$ $\langle w_{1},w_{2}\rangle \in r$ \dots $w_{2}$ je dosažitelný z $w_{1}$ 

$e: W \times V \rightarrow \{0,1\}$

$||\square\varphi||_{\mathbb{K},w} = 1$ pokud \underline{pro každý} $w'\in W$ platí: pokud $\langle w, w'\rangle\in r$ pak $||\varphi||_{\mathbb{K},w'}=1$

$||\diamond\varphi||_{\mathbb{K},w} = 1$ \dots existuje \dots

  \item Syntaktické
  \end{enumerate}
\end{enumerate}


%\newpage
\renewcommand{\indexcolumns}{3}
\printindex
\end{document}