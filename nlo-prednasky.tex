
\documentclass[10pt, a4paper, titlepage]{article}

\usepackage[utf8]{inputenc}				% napsáno v UTF-8
\usepackage[tables,figures]{upreport}	% UPOL styl
\usepackage{upstyles}
\usepackage{amsmath}
\usepackage{amssymb}
\usepackage{amsthm}
\usepackage{alltt}


\setlength{\parindent}{0pt}
\setlength{\parskip}{1ex plus 0.5ex minus 0.2ex}

\title{KMI/NLO  -- Neklasické logiky}
\author{\normalsize{Radek Janoštík (radek.janostik01@upol.cz)}}
\date{\today}

\docinfo{Radek Janoštík}{KMI/NLO  -- Neklasické logiky}

\newtheoremstyle{note}
{15pt}
{15pt}
{}
{}
{\bfseries}
{:}
{.5em}
{}
\theoremstyle{note}
\newtheorem{dukaz}{Důkaz}
\newtheorem{veta}{Věta}
\newtheorem{definice}{Definice}
\newtheorem{priklad}{Příklad}
\newtheorem{poznamka}{Poznámka}
\newtheorem{dusledek}{Důsledek}


\abstract{
Tento dokument je pouze přepisem zápisků a poznámek
z přednášek předmětu KMI/NLO. Přednášel \link{doc. Vilém Vychodil PhD}{http://vychodil.inf.upol.cz/}.}

\pagestyle{empty}
\makeindex

\begin{document}
\maketitle
\section{Přednáška 1 - jemný úvod}
\subsection{Formální logika} - studium vyplývání $\rightarrow$ formalizuje výroky, výrazy přirozeného jazyka $\rightarrow$ formule. Definuje se, že formule je/není důsledkem jiných formulí.

\subsection{Odlišnosti logik}
\begin{enumerate}
\item Co vše popisuje jazyk - tj. co jsme schopni vyjádřit pomocí formulí.

  Př.:
  \begin{enumerate}
    \item Výroková logika - zabývá se výroky - neformálně výraz, o kterém se uvažuje, že je pravdivý či ne. 

      Atomická formule - nemůže se dělit na podvýrazy pomocí spojek. Nahrazují je výrokové symboly

      Složitější formule - 1)Výrokový symbol je formule.

			   2) Je-li $\varphi$ formule, pak i $\neg \varphi$ je formule.

			   3) Jsou-li $\varphi ,\psi$ formule, pak i $\varphi \Rightarrow \psi$ je formule.
  \item Predikátová logika - zabývá se (mj.) strukturou výroků

      $(\forall x)(\forall y) (x \leq y \Rightarrow f(x) \leq f(y))$ - formule jazyka, kde $R = \{\leq\}$, $F = \{f\}$

  \item Modální logiky - formalizují modality - "muset", "moci" \dots

      Modální výroková - $\square$ \dots musí, $\diamond$ \dots může.

	Formule: Je-li $\varphi$ formule, pak i $\square\varphi$ a $\diamond\varphi$ jsou formule.

  Paradox Arnošta Večerky: \uv{Když mám 10 korun, koupím si čokoládu.}:  $\varphi \Rightarrow\psi$

  \uv{Když mám 10 korun, koupím si bonbon.}: $\varphi \Rightarrow \chi$

  $T=\{\varphi\Rightarrow\psi , \varphi\Rightarrow\chi\}$  $T \vdash \varphi\Rightarrow(\psi \wedge \chi)$
    
Modální logika dodá \uv{může}.
  $T=\{\varphi\Rightarrow\diamond\psi,\varphi\Rightarrow\diamond\chi\}$ $T\vdash\varphi\Rightarrow(\diamond\psi\wedge\diamond\chi)$. Pozor: $T \not\vdash \varphi\Rightarrow\diamond (\psi\wedge\chi)$
  \end{enumerate}

\item Tím, jak zavádí vyplývání
  \begin{enumerate}
    \item Sémantické - navrhneme interpretaci formulí.

    VL: zavedeme ohodnocení: $e: V\rightarrow\{0,1\}$ $||\varphi||_{e}$ \dots

    PL: $\langle R, F, \sigma\rangle \rightarrow \mathbb{M}=\langle M, R^{M}, F^{M}\rangle$ $||\varphi||_{M,v}$ \dots $T \models \varphi$

    mod. VL: $\square\varphi , \diamond\varphi$ - Kripkeho struktura - $\mathbb{K} = \langle W, r, e\rangle$ 

  $r\subseteq W \times W$ $\langle w_{1},w_{2}\rangle \in r$ \dots $w_{2}$ je dosažitelný z $w_{1}$ 

$e: W \times V \rightarrow \{0,1\}$

$||\square\varphi||_{\mathbb{K},w} = 1$ pokud \underline{pro každý} $w'\in W$ platí: pokud $\langle w, w'\rangle\in r$ pak $||\varphi||_{\mathbb{K},w'}=1$

$||\diamond\varphi||_{\mathbb{K},w} = 1$ \dots existuje \dots

  \item Syntaktické \dots důkaz

  kl. VL: Pravidlo: z $\varphi, \varphi\Rightarrow\psi$ odvoď $\psi$

    $(Ax): \varphi\Rightarrow(\psi\Rightarrow\varphi)$

    $(\varphi\Rightarrow(\varphi\Rightarrow\chi))\Rightarrow((\varphi\Rightarrow\psi)\Rightarrow(\varphi\Rightarrow\chi))$

    $(\neg\psi\Rightarrow\neg\varphi)\Rightarrow(\varphi\Rightarrow\psi)$

    PL: Pravidlo: z $\varphi$ odvoď $(\forall x)\varphi$

    Distrib: $(\forall x)\varphi\Rightarrow\varphi(x/t)$

    Spec: $(\forall x)(\varphi\Rightarrow\psi)\Rightarrow(\varphi\Rightarrow(\forall x) \psi)$

  \end{enumerate}

  Lze zavést vyplývání jinak? - Alternativní syntaktické vyplývání - Gentzenovské dokazovací systémy - \uv{natural deduction}

  Např.: V.Vychodil - prahová booleovská logika.
\end{enumerate}

\subsection{Co budeme zkoumat tento semestr}
Neklasické logiky, ve kterých se uvažuje, že atomické formule mohou nabývat \underline{stupňů pravdivosti} - $0$\dots$1$ - mezní dva stupně.

  $0$\dots (plně) nepravdivý

  $1$\dots (plně) pravdivý

  $0 < a < 1$ \dots stupeň pravdivosti

$\rightarrow$ \underline{Základní interpretace} $\rightarrow$ komparativní

$||\varphi||_{a} = a, ||\psi||_{b}=b$ $a \leq n$ \dots $\varphi$ je méně pravdivá než $\psi$.

$e: V\rightarrow L$ $\mathbb{L} = \langle L, \leq, 0, 1\rangle$ \dots ohraničená uspořádaná množina.

$||\varphi||_{e} \in L$

$\mathbb{L} = \langle L, \wedge, \vee, 0, 1\rangle$ \dots úplný svaz (zbytečně silné)

$\mathbb{L} = \langle L, \wedge, \vee, 0, 1\rangle$ \dots svaz (ohraničený)

Zbývá vyřešit, jak interpretovat logické spojky a které spojky vzít jako základní.

Princip kompozicionality: $||\varphi\Rightarrow\psi||_{e} = ||\varphi||_{e}\rightarrow||\psi||_{e}$

$\rightarrow$ - logická operace, která interpretuje $\Rightarrow$: 

$\rightarrow : L\times L \rightarrow L$:
 \begin{tabular}{l|*2c}
     \bf $\rightarrow$ & 0 & 1  \\
     \hline
    0 & 1& 1 \\
    1 & 0 & 1 \\
   \end{tabular}

60. léta Lotfi Askerzadeh - koncept fuzzy množiny - $\wedge\dots\min$, $\vee\dots\max$, $\neg \dots 1-a$

J.A.Goguen - $\mathbb{L} = \langle L, \wedge, \vee, 0, 1\rangle$

\underline{Modus ponens} - $\frac{\varphi\Rightarrow\psi,\varphi}{\psi}$ - 1. pohled $\frac{||\varphi\Rightarrow\psi||_{e}=1,||\varphi||_{e} = 1}{||\psi||_{e}=1}$

2. pohled - min. dolní mez pravdivosti $\psi$ odvozujeme z min. dolních mezí předchozích dvou. $\frac{a\leq||\varphi\Rightarrow\psi||, b\leq ||\varphi||}{a\otimes b\leq||\psi||}$ \dots$\otimes\rightarrow\{0,1\}\dots$ pravdivostní funkce konjunkce.

\underline{Pozorování}: $\Rightarrow\dots||\varphi\Rightarrow\psi|| = ||\varphi||\rightarrow||\psi||$

  $\otimes\dots$ pravdivostní funkce konjunkce

Jaký by měly mít vztah $(\otimes,\rightarrow)$ ? Chceme, aby zobecněné MP bylo korektní:

Pokud $a\leq||\varphi\Rightarrow\psi||$ a $b\leq||\varphi||$ pak $a\otimes b \leq||\psi||$ pro $b=||\varphi||$ a $c=||\psi||$ použitím $||\varphi\Rightarrow\psi||=||\varphi||\rightarrow||\psi||=b\rightarrow c$

Pokud $a\leq||\varphi\Rightarrow\psi|| = b\rightarrow c$, pak $a\otimes b\leq c$

$a\leq b\rightarrow c$ pak $a\otimes b \leq c$

Zobecněné MP mělo maximální možnou sílu - $b=||\varphi||, c=||\psi||$:

Pokud $a\otimes b\leq||\psi||=c$, pak $a\leq||\varphi||\rightarrow||\psi||=b\rightarrow c$

Dohromady: $$a\otimes b\leq c \text{ p.k. } a\leq b\rightarrow c \dots \text{ adjunkce}$$

(Úplný) reziduovaný svaz $\mathbb{L} = \langle L, \wedge, \vee,\otimes,\rightarrow,0,1\rangle$

$\langle L, \wedge,\vee, 0, 1\rangle \dots$(úplný) svaz

$\langle L, \otimes, 1\rangle\dots\otimes\dots$binární operace, komutativní, asociativní, $a\otimes 1=a$

$\rightarrow\dots$ binární operace, která splňuje adjunkci.

\vspace{1cm}
MP: $\frac{\varphi, \neg\varphi\vee\psi}{\psi} \rightarrow \frac{\varphi\vee\psi, \neg\varphi}{\psi}$ skrytý sém. význam: $\frac{1\leq||\varphi\vee\psi||, ||\varphi||\leq 0}{1\leq||\psi||}$

$\frac{a\leq||\varphi\vee\psi||, ||\varphi||\leq b}{c\leq||\psi||}$, \hspace{2cm} $c=a\ominus b$ \hspace{2cm}   \begin{tabular}{l|*2c}
     \bf $\ominus$ & 0 & 1  \\
     \hline
    0 & 0& 0 \\
    1 & 1 & 0 \\
   \end{tabular} $a \ominus b = \neg(a\rightarrow b)$

$\vee\dots\oplus$\hspace{3cm} $\ominus\dots$ pravdivostní funkce abjunkce - \uv{neimplikace}

Pro $b = ||\varphi||, c=||\psi||$ korektnost: $a\leq||\varphi\vee\psi|| = b\oplus c$ pak $a\ominus b\leq c$

síla: $a\ominus b \leq ||\psi||=c$ pak $a\leq b\oplus c$

$\oplus\dots$ komutativní, asociativní, neutrální vůči 0

$\ominus\dots$ adjungovaná k $\oplus$

$$a\ominus b\leq c \text{ p. k. } a\leq b \oplus c$$
\renewcommand{\indexcolumns}{3}
\printindex
\end{document}